\documentclass[spanish, 12pt, a4paper, twoside]{article}
\usepackage[utf8]{inputenc}
\usepackage[spanish]{babel}

\raggedright
\begin{document}


\title{Proyectos y Legislación}

\author{Carlos G. Pérez Aranda}
\date{2023}
\maketitle
\center{Asignatura de Proyectos y Legislación. 3º de Grado en Ingeniería Informática. Universidad de Málaga.\break}


\hrule
\raggedright
\section{Introducción}
% Remove the incomplete \left command
\begin{itemize}
\item Exámenes liberatorios al final de los temas 2,3,4 y 5.
\item Trabajos en Grupos Reducidos. 4 grupos reducidos en la clase. 2 trabajos en grupo. Grupos de 5 - 6 personas.
\item Solo uno de los miembros del grupo es el que presenta. Se llama el relator.
\item Segunda convocatoria 4 preguntas sobre la materia de la asignatura.
\item Clases en las franjas del lunes y jueves. 15:30
\end{itemize}

\section{Temas}

El temario y los materiales se encuentran en el campus virtual.\\
Los materiales incluyen los apuntes, los casos y los tests de evaluación.


\begin{enumerate}
    \item Introducción
    \item Ley de Protección de Datos
    \item Protección de las Creaciones Informáticas
    \item Regulación de Internet
    \item Proyectos y Contratos
    \item Aspectos Sociales y Económicos de la Profesión Informática (mediante trabajos)
\end{enumerate}

\subsection[short]{Introducción}

\hrule \break
\subsection{Ley de Protección de Datos}

\hrule \break
\subsection{Protección de las Creaciones Informáticas}

\hrule \break
\subsection{Regulación de Internet}

\hrule \break
\subsection{Proyectos y Contratos}

\hrule \break
\subsection{Aspectos Sociales y Económicos de la Profesión Informática}


\end{document}



\documentclass[spanish, 12pt, a4paper, twoside]{article}
\usepackage[utf8]{inputenc}
\usepackage[usenames]{color}

\begin{document}


\title{Proyectos y Legislación}

\author{Carlos G. Pérez Aranda}
\date{2023}
\maketitle{Asignatura de Proyectos y Legislación. 3º de Grado en Ingeniería Informática. Universidad de Málaga.\hfill\break}


\hrule

\section{Introducción}
% Remove the incomplete \left command
\begin{itemize}
\item Exámenes liberatorios al final de los temas 2,3,4 y 5.
\item Trabajos en Grupos Reducidos. 4 grupos reducidos en la clase. 2 trabajos en grupo. Grupos de 5 - 6 personas.
\item Solo uno de los miembros del grupo es el que presenta. Se llama el relator.
\item Segunda convocatoria 4 preguntas sobre la materia de la asignatura.
\item Clases en las franjas del lunes y jueves. 15:30
\end{itemize}

\section{Temas}

El temario y los materiales se encuentran en el campus virtual.\\
Los materiales incluyen los apuntes, los casos y los tests de evaluación.


\begin{enumerate}
    \item Introducción
    \item Ley de Protección de Datos
    \item Protección de las Creaciones Informáticas
    \item Regulación de Internet
    \item Proyectos y Contratos
    \item Aspectos Sociales y Económicos de la Profesión Informática (mediante trabajos)
\end{enumerate}
\hfill \break
\hrule
\subsection[short]{Introducción}
\textit{Ya desarrollado más arriba.}
\hfill \break
\hrule 
\subsection{Ley de Protección de Datos}


Hemos leido la sentencia del caso de la catequista chismosa.\\

¿Qué leyes regulan en España el Derecho de protección de datos?\\
\textbf{RGPD} Reglamento General de Protección de Datos.\\

Órgano encargado de hacer cumplir esta norma:\\
\textbf{Agencia Española de Protección de Datos} AEPD.\\

Las resoluciones de la AEPD se pueden recurrir ante la Audiencia Nacional y en última instacia 
ante el Tribunal Supremo.

\hfill\break
Tipos de sanciones:
\begin{itemize}
    \item apercibimiento
    \item multa. Hasta 20M € o el 4\% de la facturación anual global.
    \item limitación / prohibición de tratamiento
\end{itemize}


La AEPD no puede ordenar que se pague una compensación a una persona lesionada en sus derechos. Las multas
impuestas tienen el carácter desanción administrativa e integra el Tesoro Público.
El resarcimineto de daños y erjuicios debe ser reclamado en la vía civil. A este respecto es especialmente relevante la 
Ley Orgánica 1/1982 de 5 de mayo de protección civil del derecho al honor, a la intimidad personal y familiar y a la propia imagen.\\

Si los daños los causó un organismo públixo, deberá exigir la responsabilidad patrimonial del Estado ante los correspondientes órganos
administrativos o recurrir a la jurisdicción contencioso-administrativa.\\


La AEPD es un órgano administrativo, por lo tanto no puede condenar a penas de cárcel a los que infrinjan los preceptos del RGPD.
Esto es competencia exclusiva de los tribunales del orden penal.\\


La difusión de imágenes íntimas está tipificada como delito siempre que la grabación se realice con el permiso 
de la persona afectada en un domicilio o en cualquier lugar fuera del alcance de terceros, cuando la divulgación menoscabe gravemente la intimidad personal del afectado. \textbf{Artículo 197.7 del Código Penal}\\

En el sitio web de la AEPD www.aepd.es hay abundante información.

\subsubsection{Tratamiento de datos}

Se entiende por \textit{datos personales} toda información sobre una persona física identificada o identificable.
Se considerará persona física identificable toda persona cuya identidad pueda determinarse, directa o 
indirectamente, en particular mediante un identificador, como por ejemplo un nombre, un número de identificación
, datos de localización, un identificados en línea o uno o varios propios de la identidad f´ısica, fisiol´ogica, gen´etica, ps´ıquica, econ´omica, cultural
o social de dicha persona.\\

\textbf{Seudonimización}. El tratamiento de dates personales de manera qye ya no puedan atribuirse a un interesado
sin utilizar información adicional, siempre que dicha información adicional se mantenga separada y sujeto 
a medidas técnicas y organizativas destinadas a garantizar que los datos personales no se atribuyam a una persona física
identificada o identificable.\\

Los datos seudonimizados no son datos anónimos. Información \textbf{anónima} es aquella que no puede ser asociada a una persona
 física identificada o identificable. El RGPD no se aplica a la información anónima, inclusive con fines
 estadísticos o de investigación, pero sí al tratamiento de los datos seudonimizados\\

Son datos personales:
\begin{itemize}
    \item Las direcciones de correos privadas.
    \item Las direcciones IP.
    \item Imágenes del rostro
\end{itemize}

Hay categorías \textbf{especiales} de datos personales que requieren un tratamiento especial.\\

\begin{itemize}
    \item Que revelen el origen étnico o racial, las opiniones políticas, las convicciones
    religiosas o filosóficas, o la afiliación sindical;
    \item Genéticos y biométricos dirigidos a identificar de manera unívoca a una
persona física;
    \item Relativos a la salud.
    \item Relativos a la vida sexual o la orientación sexual de una persona.
\end{itemize}
El tratamiento de estos datos está sometido a restricciones adicionales.\\

El tratamiento de datos es Cualquier operacoón o conjunto de operaciones realizadas sobre datos perso-
nales o conjuntos de datos personales, ya sea por procedimientos automatiza-
dos o no, como la recogida, registro, organización, estructuración, conservación,
adaptación o modificación, extracción, consulta, utilización, comunicación por
transmisión, difusión o cualquier otra forma de habilitación de acceso, cotejo o
interconexión, limitación, supresión o destrucción.\\

Los tratamientos de datos que son objeto de aplicación del RGPD son los total o percialmente
\textbf{automatizados y los no automatizados} que formen parte de un fichero o sean susceptibles de ser
Se entiende por \textit{fichero} cualquier conjunto estructurado de datos personales accesibles según 
criterios determinados, ya sea centralizado, descentralizado o repartido de forma funcional o geográfica.\\

\textbf{Excepciones:}
\begin{itemize}
    \item El tratamiento efectuado por una persona física en el ejercicio de actividades exclusivamente
    personales o domésticas.
    \item El tratamiento de datos efectuado por las autoridades competentes con fines de prevención, investigación,
\end{itemize}
\break

\textbf{Roles en el tratamiento de datos:}\\
\begin{itemize}
    \item \textbf{Responsable del tratamiento}. La persona física o jurídica, autoridad pública, servicio u otro organismo que, solo o junto con otros, determine los fines y medios del tratamiento.
    \item \textbf{Interesado}. La persona física cuyos datos son objeto de tratamiento.
    \item \textbf{Encargado del tratamiento}. La persona física o jurídica, autoridad pública, servicio u otro organismo que trate datos personales por cuenta del responsable del tratamiento.
\end{itemize}

\hfill \break 

\textbf{Ejemplos}
\textit{Ejemplos como estos caerán en el examen.}
\begin{itemize}
    \item \textbf{Cuestión 1.} La Universidad de Villalquite tiene como Rector a don Numerio Negidio. Dentro de la Universidad existe un Servicio de Informática, responsable entre otras cosas del procesamiento de los datos de matrícula de los alumnos. La Directora del Servicio es doña Lucía Ticia.
    Indicar, caso de que existan, quiénes son (i) el o los interesados en el tratamiento de los datos de matrícula; (ii) el o los responsables del tratamiento; (iii) el o los encargados del tratamiento.
    \begin{enumerate}
        \item Interesados: Los alumnos que se matriculan.
        \item Responsable del tratamiento: Universidad de Villalquite.
        \item Encargado del tratamiento: No hay encargado del tratamiento, ya que, por definición, el encargado del tratamiento debe ser subcontratado para que exista.
    \end{enumerate}

    \item \textbf{Cuestión 2.} Tras acabar sus estudios en la Universidad de Villalquite, don Estico Estíquez trabaja como asesor fiscal autónomo. Uno de sus clientes es la empresa “Kkfashion, Escuela de modelos”, que le tiene encomendada la gestión de su contabilidad (facturas, clientes, proveedores, alumnos, . . . )
    Indicar, caso de que existan, quiénes son (i) el o los interesados en el tratamiento de los datos de contabilidad; (ii) el o los responsables del tratamiento; (iii) el o los encargados del tratamiento.
    \begin{enumerate}
        \item Interesados: Los clientes personas físicas de la empresa “Kkfashion, Escuela de modelos”.
        \item Responsable del tratamiento: “Kkfashion, Escuela de modelos”.
        \item Encargado del tratamiento: Estico Estíquez.
    \end{enumerate}
\end{itemize}
    
\subsubsection{Principios del tratamiento.}

\begin{enumerate}
    \item \textbf{Limitación de la Finalidad.} Los datos serán recogidos con fines determinados, explícitos y legítimos, y no serán tratados ulteriormente de manera incompatible con dichos fines. El tratamiento ulterior de los datos personales con fines de archivo en interés público, fines de investigación científica e histórica o fines estadísticos no se consideraría incompatible con los fines iniciales.
    \item \textbf{Minimización de datos.} Los datos personales serán adecuados, pertinentes y limitados a lo necesario en relación con los fines para los que son tratados.
    \item \textbf{Exactitud.} Los datos personales serán exactos y, si fuera necesario, actualizados; se adoptarán todas las medidas razonables para que se supriman o rectifiquen sin dilación los datos personales que sean inexactos con respecto a los fines para los que se tratan.
    Es ejemplo de este último punto las sanciones interpuestas por apariciones en los archivos de morosos sin ser deudor o sin poder demostrar la existencia de tal deuda.
    \item \textbf{Limitación del plazo de conservación.} Los datos personales serán mantenidos de forma que se permita la identificación de los interesados durante no más tiempo del necesario para los fines del tratamiento de los datos personales. Podrán conservarse durante periodos más largos siempre que se traten exclusivamente con fines de archivo en interés público, fines de investigación científica e histórica o fines estadísticos.
    \item \textbf{Integridad y confidencialidad.} Los datos personales serán tratados de tal manera que se garantice una seguridad adecuada de los datos personales, incluida la protección contra el tratamiento no autorizado o ilícito y contra su pérdida, destrucción o daño accidental, mediante la aplicación de medidas técnicas u organizativas apropiadas. Todas las personas que intervengan en esta fase del tratamiento estarán sujetas al principio de confidencialidad.
    \item \textbf{Legitimación del tratamiento.} El tratamiento de datos personales solo será lícito si se cumple al menos una de las causas o bases de legitimacióne enumeradas en el RGPD.    
    \item \textcolor{red}{\textbf{Responsabilidad proactiva.}} El responsable del tratamiento será responsable del cumplimiento de los principios del tratamiento anteriores y debe poder demostrarlo.
\end{enumerate}

\subsubsection{Bases de legitimación enumeradas en el RGPD}

\begin{itemize}
    \item \textbf{Consentimiento del interesado.} El interesado ha dado su consentimiento para el tratamiento de sus datos personales para uno o varios fines específicos. \textbf{Consentimiento } es toda manifestación libre, específica, informada e inequívoca por la que el interesado por la que el la que el interesado acepta, ya sea mediante una declaración o una clara acción afirmativa, el tratamiento de datos personales que le conciernen. El responsable deberá ser capaz de demostrar que el interesado consintió el tratamiento de sus datos personales. Nótese que se exige una declaración o una “acción afirmativa”; el consentimiento no se supone, ni puede ser tácito.
    
    El RGPD da indicaciones sobre como \textbf{redactar la solicitud del consentimiento.}


    Si el consentimiento se da en el contexto de una declaración escrita que también se refiera a otros asuntos, 
    la solicitud de consentimiento se presentará de tal forma que se 
    distinga claramente de los demás asuntos, de forma inteligible y 
    de fácil acceso y utilizando un lenguaje claro y sencillo.
    Además, para evaluar si el consentimiento se ha dado libremente 
    se tendrá en cuenta el hecho de si, entre otras cosas, la ejecución
    de un contrato se supedita al consentimiento al tratamiento de datos
    personales que no son necesarios para la ejecución de dicho 
    contrato.
    \item \textbf{Interés Legítimo del Responsable (o un 3º)}
    \item \textbf{Cumplimiento de una obligación legal.}
    \item \textbf{Protección de intereses vitales del interesado o de otra persona física.}
    \item \textbf{Ejecución de un contrato.}
    \item \textbf{Ejercicio de poderes públicos.}
    \item 
\end{itemize}
\hfill \break
\newpage
\hrule
\subsection{Protección de las Creaciones Informáticas}

\newpage
\hrule
\subsection{Regulación de Internet}

\newpage
\hrule
\subsection{Proyectos y Contratos}

\newpage
\hrule
\subsection{Aspectos Sociales y Económicos de la Profesión Informática}


\end{document}



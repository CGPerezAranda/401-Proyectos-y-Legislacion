\documentclass[spanish, 12pt, a4paper, twoside]{article}
\usepackage[utf8]{inputenc}


\begin{document}


\title{Proyectos y Legislación}

\author{Carlos G. Pérez Aranda}
\date{2023}
\maketitle{Asignatura de Proyectos y Legislación. 3º de Grado en Ingeniería Informática. Universidad de Málaga.\break}


\hrule

\section{Introducción}
% Remove the incomplete \left command
\begin{itemize}
\item Exámenes liberatorios al final de los temas 2,3,4 y 5.
\item Trabajos en Grupos Reducidos. 4 grupos reducidos en la clase. 2 trabajos en grupo. Grupos de 5 - 6 personas.
\item Solo uno de los miembros del grupo es el que presenta. Se llama el relator.
\item Segunda convocatoria 4 preguntas sobre la materia de la asignatura.
\item Clases en las franjas del lunes y jueves. 15:30
\end{itemize}

\section{Temas}

El temario y los materiales se encuentran en el campus virtual.\\
Los materiales incluyen los apuntes, los casos y los tests de evaluación.


\begin{enumerate}
    \item Introducción
    \item Ley de Protección de Datos
    \item Protección de las Creaciones Informáticas
    \item Regulación de Internet
    \item Proyectos y Contratos
    \item Aspectos Sociales y Económicos de la Profesión Informática (mediante trabajos)
\end{enumerate}
\hfill \break
\hrule
\subsection[short]{Introducción}
\textit{Ya desarrollado más arriba.}
\hfill \break
\hrule 
\subsection{Ley de Protección de Datos}

Hemos leido la sentencia del caso de la catequista chismosa.

¿Qué leyes regulan en España el Derecho de protección de datos?\\
\textbf{RGPD} Reglamento General de Protección de Datos.

Órgano encargado de hacer cumplir esta norma:\\
\textbf{Agencia Española de Protección de Datos} AEPD.\\
Las resoluciones de la AEPD se pueden recurrir ante la Audiencia Nacional y en última instacia 
ante el Tribunal Supremo.

\hfill\break
Tipos de sanciones:
\begin{itemize}
    \item apercibimiento
    \item multa. Hasta 20M € o el 4\% de la facturación anual global.
    \item limitación / prohibición de tratamiento
\end{itemize}


La AEPD no puede ordenar que se pague una compensación a una persona lesionada en sus derechos. Las multas
impuestas tienen el carácter desanción administrativa e integra el Tesoro Público.
El resarcimineto de daños y erjuicios debe ser reclamado en la vía civil. A este respecto es especialmente relevante la 
Ley Orgánica 1/1982 de 5 de mayo de protección civil del derecho al honor, a la intimidad personal y familiar y a la propia imagen.\\

Si los daños los causó un organismo públixo, deberá exigir la responsabilidad patrimonial del Estado ante los correspondientes órganos
administrativos o recurrir a la jurisdicción contencioso-administrativa.\\


La AEPD es un órgano administrativo, por lo tanto no puede condenar a penas de cárcel a los que infrinjan los preceptos del RGPD.
Esto es competencia exclusiva de los tribunales del orden penal.\\


La difusión de imágenes íntimas está tipificada como delito siempre que la grabación se realice con el permiso 
de la persona afectada en un domicilio o en cualquier lugar fuera del alcance de terceros, cuando la divulgación menoscabe gravemente la intimidad personal del afectado. \textbf{Artículo 197.7 del Código Penal}\\

En el sitio web de la AEPD www.aepd.es hay abundante información.

\subsubsection*{Tratamiento de datos}

\hfill \break
\newpage
\hrule
\subsection{Protección de las Creaciones Informáticas}

\newpage
\hrule
\subsection{Regulación de Internet}

\newpage
\hrule
\subsection{Proyectos y Contratos}

\newpage
\hrule
\subsection{Aspectos Sociales y Económicos de la Profesión Informática}


\end{document}


